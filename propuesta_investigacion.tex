
%-----------------------------------------------------------------------------------------------------------------------------------------------%
%	The MIT License (MIT)
%
%	Copyright (c) 2019 Jan Küster
%
%	Permission is hereby granted, free of charge, to any person obtaining a copy
%	of this software and associated documentation files (the "Software"), to deal
%	in the Software without restriction, including without limitation the rights
%	to use, copy, modify, merge, publish, distribute, sublicense, and/or sell
%	copies of the Software, and to permit persons to whom the Software is
%	furnished to do so, subject to the following conditions:
%	
%	THE SOFTWARE IS PROVIDED "AS IS", WITHOUT WARRANTY OF ANY KIND, EXPRESS OR
%	IMPLIED, INCLUDING BUT NOT LIMITED TO THE WARRANTIES OF MERCHANTABILITY,
%	FITNESS FOR A PARTICULAR PURPOSE AND NONINFRINGEMENT. IN NO EVENT SHALL THE
%	AUTHORS OR COPYRIGHT HOLDERS BE LIABLE FOR ANY CLAIM, DAMAGES OR OTHER
%	LIABILITY, WHETHER IN AN ACTION OF CONTRACT, TORT OR OTHERWISE, ARISING FROM,
%	OUT OF OR IN CONNECTION WITH THE SOFTWARE OR THE USE OR OTHER DEALINGS IN
%	THE SOFTWARE.
%	
%
%-----------------------------------------------------------------------------------------------------------------------------------------------%


%============================================================================%
%
%	DOCUMENT DEFINITION
%
%============================================================================%

%we use article class because we want to fully customize the page and don't use a cv template
\documentclass[12pt,letter,spanish]{article}	


%----------------------------------------------------------------------------------------
%	ENCODING
%----------------------------------------------------------------------------------------

% we use utf8 since we want to build from any machine
\usepackage[utf8]{inputenc}		
\usepackage[spanish]{isodate}
\usepackage{fancyhdr}
\usepackage[numbers]{natbib}
\usepackage{mhchem}  % Use chemical formulas  Example: \ce{AgSbS_2}

%----------------------------------------------------------------------------------------
%	LOGIC
%----------------------------------------------------------------------------------------

% provides \isempty test
\usepackage{xstring, xifthen}
\usepackage{enumitem}
\usepackage[spanish]{babel}
\usepackage{lipsum}
\usepackage{pdfpages}
\usepackage{changepage}

%----------------------------------------------------------------------------------------
%	FONT BASICS
%----------------------------------------------------------------------------------------

% some tex-live fonts - choose your own

%\usepackage[defaultsans]{droidsans}
%\usepackage[default]{comfortaa}
%\usepackage{cmbright}
\usepackage[default]{raleway}
%\usepackage{fetamont}
%\usepackage[default]{gillius}
%\usepackage[light,math]{iwona}
%\usepackage[thin]{roboto} 

% set font default
\renewcommand*\familydefault{\sfdefault} 	
\usepackage[T1]{fontenc}

% more font size definitions
\usepackage{moresize}

%----------------------------------------------------------------------------------------
%	FONT AWESOME ICONS
%---------------------------------------------------------------------------------------- 

% include the fontawesome icon set
\usepackage{fontawesome}

% use to vertically center content
% credits to: http://tex.stackexchange.com/questions/7219/how-to-vertically-center-two-images-next-to-each-other
\newcommand{\vcenteredinclude}[1]{\begingroup
\setbox0=\hbox{\includegraphics{#1}}%
\parbox{\wd0}{\box0}\endgroup}
\newcommand{\tab}[1]{\hspace{.2\textwidth}\rlap{#1}}
% use to vertically center content
% credits to: http://tex.stackexchange.com/questions/7219/how-to-vertically-center-two-images-next-to-each-other
\newcommand*{\vcenteredhbox}[1]{\begingroup
\setbox0=\hbox{#1}\parbox{\wd0}{\box0}\endgroup}

% icon shortcut
\newcommand{\icon}[3] { 							
	\makebox(#2, #2){\textcolor{maincol}{\csname fa#1\endcsname}}
}	


% icon with text shortcut
\newcommand{\icontext}[4]{ 						
	\vcenteredhbox{\icon{#1}{#2}{#3}}  \hspace{2pt}  \parbox{0.9\mpwidth}{\textcolor{#4}{#3}}
}

% icon with website url
\newcommand{\iconhref}[5]{ 						
    \vcenteredhbox{\icon{#1}{#2}{#5}}  \hspace{2pt} \href{#4}{\textcolor{#5}{#3}}
}

% icon with email link
\newcommand{\iconemail}[5]{ 						
    \vcenteredhbox{\icon{#1}{#2}{#5}}  \hspace{2pt} \href{mailto:#4}{\textcolor{#5}{#3}}
}

%----------------------------------------------------------------------------------------
%	PAGE LAYOUT  DEFINITIONS
%----------------------------------------------------------------------------------------

% page outer frames (debug-only)
% \usepackage{showframe}		

% we use paracol to display breakable two columns
\usepackage{paracol}
\usepackage{tikzpagenodes}
\usetikzlibrary{calc}
\usepackage{lmodern}
\usepackage{multicol}
\usepackage{lipsum}
\usepackage{atbegshi}
% define page styles using geometry
\usepackage[a4paper]{geometry}

% remove all possible margins
\geometry{top=1cm, bottom=1cm, left=1cm, right=1cm}

\usepackage{fancyhdr}
\pagestyle{empty}

% space between header and content
% \setlength{\headheight}{0pt}

% indentation is zero
\setlength{\parindent}{0mm}

%----------------------------------------------------------------------------------------
%	TABLE /ARRAY DEFINITIONS
%---------------------------------------------------------------------------------------- 

% extended aligning of tabular cells
\usepackage{array}

% custom column right-align with fixed width
% use like p{size} but via x{size}
\newcolumntype{x}[1]{%
>{\raggedleft\hspace{0pt}}p{#1}}%


%----------------------------------------------------------------------------------------
%	GRAPHICS DEFINITIONS
%---------------------------------------------------------------------------------------- 

%for header image
\usepackage{graphicx}

% use this for floating figures
% \usepackage{wrapfig}
% \usepackage{float}
% \floatstyle{boxed} 
% \restylefloat{figure}

%for drawing graphics		
\usepackage{tikz}			
\usepackage{ragged2e}	
\usetikzlibrary{shapes, backgrounds,mindmap, trees}

%----------------------------------------------------------------------------------------
%	Color DEFINITIONS
%---------------------------------------------------------------------------------------- 
\usepackage{transparent}
\usepackage{color}

% primary color
\definecolor{maincol}{RGB}{ 64,64,64}

% accent color, secondary
% \definecolor{accentcol}{RGB}{ 250, 150, 10 }

% dark color
\definecolor{darkcol}{RGB}{ 70, 70, 70 }

% light color
\definecolor{lightcol}{RGB}{245,245,245}

\definecolor{accentcol}{RGB}{59,77,97}



% Package for links, must be the last package used
\usepackage[hidelinks]{hyperref}

% returns minipage width minus two times \fboxsep
% to keep padding included in width calculations
% can also be used for other boxes / environments
\newcommand{\mpwidth}{\linewidth-\fboxsep-\fboxsep}
	

%----------------------------------------------------------------------------------------
%	 CV TEXT
%----------------------------------------------------------------------------------------

% base class to wrap any text based stuff here. Renders like a paragraph.
% Allows complex commands to be passed, too.
% param 1: *any
\newcommand{\cvtext}[1] {
	\begin{tabular*}{1\mpwidth}{p{0.98\mpwidth}}
		\parbox{1\mpwidth}{#1}
	\end{tabular*}
}



% HEADER AND FOOOTER 
%====================================


\newcommand\Header[1]{%
\begin{tikzpicture}[remember picture,overlay]
\fill[accentcol]
  (current page.north west) -- (current page.north east) --
  ([yshift=50pt]current page.north east|-current page text area.north east) --
  ([yshift=50pt,xshift=-3cm]current page.north|-current page text area.north) --
  ([yshift=10pt,xshift=-5cm]current page.north|-current page text area.north) --
  ([yshift=10pt]current page.north west|-current page text area.north west) -- cycle;
\node[font=\sffamily\bfseries\color{white},anchor=west,
  xshift=0.7cm,yshift=-0.32cm] at (current page.north west)
  {\fontsize{12}{12}\selectfont {#1}};
\end{tikzpicture}%
}
\newcommand\Footer[1]{%
\begin{tikzpicture}[remember picture,overlay]
\fill[lightcol]
  (current page.south east) -- (current page.south west) --
  ([yshift=-80pt]current page.south east|-current page text area.south east) --
  ([yshift=-80pt,xshift=-6cm]current page.south|-current page text area.south) --
  ([xshift=-2.5cm,yshift=-10pt]current page.south|-current page text area.south) --
  ([yshift=-10pt]current page.south east|-current page text area.south east) -- cycle;
\node[yshift=0.32cm,xshift=9cm] at (current page.south) {\fontsize{10}{10}\selectfont \setcounter{page}{#1}
 \textbf{}};
\end{tikzpicture}%
}

%=====================================
%============================================================================%
%
%
%
%	DOCUMENT CONTENT
%
%
%
%============================================================================%
\begin{document}

%\setlength{\columnsep}{2.2em}
%\setlength{\columnseprule}{4pt}
%\colseprulecolor{white}

\AtBeginShipoutFirst{\Header{Propuesta Investigación}\Footer{1}}
\newpage

%%%%ANSCHREIBEN

%\raggedright
\begin{center}
\large \textbf{Perovskite inspired materials for solar energy conversion}

\large Materiales Inspirados en Perovskitas para la Conversión de Energía
\end{center}


\section*{Introduccion}

Los semiconductores basados en calcogenuros metalicos han surgido como materiales clave para aplicaciones de converion de energia como la fotovoltaica(PV) y celdas fotoelectroquimicas (PEC). Estos materiales, debido a su estructura compleja, enfrentan desafíos relacionados con el desorden catiónico, históricamente visto como perjudicial para el transporte de cargas. Sin embargo, investigaciones recientes han demostrado que este desorden, si se maneja adecuadamente, puede mejorar significativamente la movilidad de los portadores de carga, lo que es crucial para el desarrollo de dipositivos semiconductores eficientes [1].\\

La eficiencia de los semiconductores convencionales ha sido limitada por la necesidad de minimizar los defectos mediante métodos costosos y de alta temperatura. Sin embargo, la reciente investigación en materiales inspirados en perovskitas, como el \ce{AgBiS2} y \ce{NaSbS2} de estructura cristalina cubica, ha mostrado que es posible lograr eficiencias del 9\% en celdas solares incluso en presencia de defectos, gracias a la “tolerancia a defectos” [2]. En nuestras investigaciones anteriores, hemos demostrado que la selenización del \ce{AgSbS2} mejora las propiedades optoelectrónicas del material semiconductor, incrementando la densidad de portadores de carga y reduciendo el ancho de banda prohibida (Eg) lo cual permite obtener celdas solares de película delgada con factores de forma del 60 \% Voc de mas de 500 mV y densidad de corriente electrica de 2 mA/cm^2 [3]. \\

En el presente proyecto, propongo investigar y desarrollar materiales inorganico inspirados en perovskitas (perovskite inspired materials), específicamente modificar el material que he desarrollado durante los ultimos años \ce{AgSbS2} para obtener \ce{AgBiS2} en una primer etapa y  continuar con \ce{NaBiS2} en una segunda etapa. Con el objetivo de crear apliciones PV y PEC que mantengan alta eficiencia en condiciones de síntesis económicas y escalables. Este proyecto se enfocará en la sintesis, caracterizacion de nuevos materiales y estudio del desorden catiónico para su aplicación en dispositivos de conversión de energía. La linea de investigacion que propongo iniciar no solo contribuirá al avance científico, sino que también tendrá un impacto tangible en la sociedad mediante la transferencia de tecnología y la formación de nuevos investigadores en un entorno interdisciplinario.

\section*{Objetivos}

\subsection*{Objetivo General}
Desarrollar y caracterizar materiales inspirados en perovskitas que toleren defectos y mantengan un rendimiento eficiente en dispositivos de conversión de energía.
\subsection*{Objetivo Especificos}
	
	1.	Sintetizar y caracterizar semiconductores inorganicos inspirados en perovskitas(\ce{AgSbS2}, \ce{AgBiS2}, \ce{NaBiS2}) para su aplicación en dispositivos de conversión de energía \\
	2.	Formar nuevos investigadores en técnicas avanzadas de síntesis y caracterización de materiales. \\
	3.	Transferir el conocimiento obtenido al desarrollo de dispositivos de conversion de energia PV y PEC mediante aspectos de innovacion tecnologica.

\section*{Antecedentes}

El sulfuro elenuro de antimonio y plata (\ce{AgSb(S,Se)2}) es un calcogenuro con una estructura cristalina cúbica y un alto coeficiente de absorción óptica ($\alpha > 10^4 \, \text{cm}^{-1}$}) en la región visible a infrarrojo cercano (VIS-NIR) del espectro electromagnético. El encho de banda prohibida (Eg) de las películas de \ce{AgSb(S,Se)2} es ajustable dentro del rango de 1.11 a 1.72 eV, y la densidad de portadores de carga de huecos ($p_p$) también se puede modificar de $10^{12}$ a $10^{18} \, \text{cm}^{-3}$. Este control sobre las propiedades se ha logrado modulando la relación atómica S:Se durante el proceso de depósito o a través de una proceso de selenización post deposito. Las excepcionales propiedades optoelectrónicas exhibidas por \ce{AgSb(S,Se)2} lo posicionan como un material fotovoltaico emergente basado en antimonio prometedor para el desarrollo de celdas solares de película delgada. Los recientes avances en la eficiencia de conversión fotovoltaica de materiales inspirados en perovskitas \ce{ABZ2}, en particular \ce{AgBiS2}, con una eficiencia de conversión registrada ($\eta$) del 9.17\%, han incrementado el interés en \ce{AgSb(S,Se)2}. En la estructura de los materiales \ce{ABZ2}, A denota un catión con un estado iónico 1+ estable, B es un catión con un estado iónico 3+ típicamente proveniente de elementos semimetálicos como el antimonio (Sb) o el bismuto (Bi), y Z es un anión con un estado iónico 2- de elementos calcogenuros como el azufre (S) o el selenio (Se).

\section*{Metodologia}

1.	Síntesis y Caracterización de Materiales: Se utilizaran tecnicas de deposito quimico para la sintesis de \ce{AgSbS2} \ce{AgBiS2} y una combinación de técnicas de síntesis, evaporacion termica, sputtering y ALD para producir materiales de alta calidad para el desarrollo de lo dispositivos de conversion de energía. La caracterización se realizara con los equipos disponible en Cinvetav Unidad Merida.
2.	Estudio de la Tolerancia a Defectos: Se investigará cómo la presencia de defectos afecta el transporte de carga y las propiedades optoelectrónicas. Este estudio nos permitirá optimizar las condiciones de síntesis para maximizar la eficiencia de dispositivos de conversión de energía.
3.	Desarrollo de Dispositivos: Aplicaremos el conocimiento adquirido para diseñar dispositivos fotovoltaicos y fotoelectroquímicos utilizando materiales no tóxicos y sostenibles. Estos dispositivos serán evaluados en términos de eficiencia energética y estabilidad a largo plazo.


\section*{Impactos}

Este proyecto tiene el potencial de revolucionar el campo de los materiales energéticos mediante la introducción de una nueva generación de semiconductores que combinen eficiencia y sostenibilidad. Los resultados de esta investigación no solo contribuirán al avance del conocimiento científico, sino que también tendrán aplicaciones prácticas en la industria de las energías renovables. Además, el proyecto servirá como una plataforma para la formación de nuevos investigadores en un entorno interdisciplinario, fomentando la transferencia de tecnología desde el laboratorio hasta la sociedad.
	
\section*{Colaboraciones}

La naturaleza interdisciplinaria de este proyecto se verá fortalecida por colaboraciones con grupos teóricos y experimentales, lo que permitirá una comprensión más profunda de las interacciones entre portadores de carga y materia. A largo plazo, buscare escalar nuestras innovaciones hacia aplicaciones industriales, contribuyendo al desarrollo de soluciones energéticas que respondan a las necesidades globales de sostenibilidad.

\section*{Bibliografia}

[1] M. Righetto et al., Cation‐Disorder Engineering Promotes Efficient Charge‐Carrier Transport in \ce{AgBiS2} Nanocrystal Films, Advanced Materials (2023) 2305009. https://doi.org/10.1002/adma.202305009.

[2] Y. Wang et al., Cation disorder engineering yields \ce{AgBiS2} nanocrystals with enhanced optical absorption for efficient ultrathin solar cells, Nat. Photon. (2022). https://doi.org/10.1038/s41566-021-00950-4.

[3] J. Capistrán-Martínez, M.T.S. Nair, P.K. Nair, Silver Antimony Sulfide Selenide Thin‐Film Solar Cells via Chemical Deposition, Phys. Status Solidi A 218 (2021) 2100058. https://doi.org/10.1002/pssa.202100058.







% ZUSAMMENFASSUNG

\end{document}